\section{Method}
\label{sec:method}

\subsection{The Neo4j database}

\subsection{Crawling Twitter}

\subsection{Parsing tweets and extracting topics}

The goal of the project is to recommend users given topics. In order to
recommend a user, the user needs to be associated with the topics the user talks
about. Therefore the users tweets are parsed and the topics of the tweets are
extracted.  The topics are extracted by parsing the freetext of the tweets and
extracting the nouns and adjectives.  The choice of extracting nouns and
adjectives was an empiric decision made by the group.

Extracting topics from tweets is done using the Natural Language Toolkit (NLTK)
\cite{bird2006nltk} which provides interfaces in Python for things like
classification, tokenization and stemming.

\subsubsection{Cleaning tweets}

A tweet can contain hyperlinks, hashtags, mentions and other symbols. These are
removed in order to properly parse the text of the tweet. Specifically, words
starting with \textit{\#, @, \& or http} are ignored. A few other words that
commonly occur in a tweet were also ignored as they would not contribute to the
cause. These are \textit{don't, i'll, retweet and rt}.

\subsubsection{Extracting nouns}

The nouns (topics) are extracted by performing the following actions, provided
by NLTK.
\begin{enumerate}
    \item Tokenize the freetext and convert them into lowercase.
    \item Remove words that are shorter than three characters. This was also a
        decision made by the group.
    \item For each word, remove ignored symbols.
    \item Skip words that start with one of the ignored symbols.
    \item Part of Speech-tag \cite{pos} the words.
    \item Pick the words that are tagged as \texttt{NN} (noun) or \texttt{JJ}
        (adjective).
    \item Stem the words and return the result.
\end{enumerate}


\subsection{Ranking, PageRank and tf-idf}

\subsection{Graphical user interface}
