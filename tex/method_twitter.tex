\subsection{Crawling Twitter}

The Twitter API \cite{twitterapi} was used to fetch tweets related to a limited
set of hashtags (Appendix \ref{appendix:hashtags}). The reason behind this constraint
was the fact that gathering generic data would require a lot more tweets to
give reasonable results. The crawler collected 10Gb of data which the group 
decided was sufficient for the use-case.

% In order to generate a significant dataset for our queries, we had to design a
% Python script which acts as a \emph{crawler}, going through Twitter data and
% storing it. That is done through the Twitter API \cite{twitterapi}. To do that,
% one has to sign up as a developer in Twitter and obtain client credentials so
% that access to the API is granted to the app. The first step of the crawler
% script is to use these credentials so as to obtain an access token.

% With access granted, our application is able to execute HTTP requests by means
% of the Requests library for python \cite{pyreq}. There are many possible
% requests offered by the Twitter API but we only used one such request, which
% queries the newest 100 tweets containing at least one of the hashtags given as
% input. It has the form:
% \emph{https://api.twitter.com/1.1/search/tweets.json?q=has\newline
% htags\&count=100\&result\_type="recent"\&lang="en"}, where "hashtags" is a
%   string containing all the hashtags we decided were relevant for our USA
%   elections context.

% So the basic working of the script consists of an infinite loop where this
% request is made and the returned \emph{json} is parsed and interpreted by a set
% of functions which add \emph{Tweets}, \emph{Users}, \emph{Hashtags} and
% \emph{Words} to our database in a recursive way (i.e. if one tweet retweets a
% tweet that mentions a user, all this data is going to be properly processed and
% stored). Also, proper term extraction and stemming is done, before storing
% \emph{Words}.
