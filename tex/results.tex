\section{Experimental results}

In both of our experiments, a toy query, "usa election president politics " was used. We first tried 
only TF-IDF and then a combined retrieval with rankType = TODO, both of which were compared to combined 
retrieval with rankType = TODO. We graded the top ten results of each retrieval as relevant or not and 
plotted a precision vs recall graph for each of them.

\subsection{TF-IDF only}

\begin{figure}[H]
\centering
\includegraphics[width=5.5in,natwidth=534,natheight=345]{images/exptfidf.png}
\caption{Precision vs Recall for TF-IDF only (left) and combination with $\alpha = 0.5$ and rankType = TODO (right).}
\label{fig:exptfidf}
\end{figure}

As expected, a combination of TF-IDF with PageRank performs better than just the first algorithm.
The problem is that our database does not contain \emph{all} the tweets of each user, so the bag
of words model is only precise among the crawled topics. If, for instance, a user posts a single tweet
about politics and we only crawl that tweet for that user, it's going to appear really well-ranked for
TF-IDF while, in reality, it shouldn't. When PageRank is added to the equation we exploit the
graph structure of Twitter and take the authority of the aforementioned user in consideration, which
highly improves our Precision vs Recall curve, as can be seen in Figure \ref{fig:exptfidf}.

\subsection{PageRank variant}

\begin{figure}[H]
\centering
\includegraphics[width=5.5in,natwidth=966,natheight=420]{images/exptopr.png}
\caption{Precision vs Recall for PageRank using the classic algorithm (left) and the one based on the number of followers.}
\label{fig:exptopr}
\end{figure}

In Figure \ref{fig:exptopr} we use two different methods for PageRank: the classic method where the jump between mentions of a user is done with equal probabilities, and using higher chance of jumping based on the number of followers. As we can see, the precision-at-10 of our novel method is considerably better than using the classic algorithm, and we attribute that to the nature of our problem: since our engine is recommending users to follow, users with high number of followers can be considered as good candidates. However, using this extensively may result in recommending the same top users always, which may end up penalizing too much relevant but unpopular users or the ones which are new in Twitter.

\subsection{Evaluation}
Summary of what alpha should be and why.
